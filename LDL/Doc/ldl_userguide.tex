\documentclass[12pt]{article}

\newcommand{\m}[1]{{\bf{#1}}}       % for matrices and vectors
\newcommand{\tr}{^{\sf T}}          % transpose

\topmargin 0in
\textheight 9in
\oddsidemargin 0pt
\evensidemargin 0pt
\textwidth 6.5in

%-------------------------------------------------------------------------------
\begin{document}
%-------------------------------------------------------------------------------

\title{User Guide for LDL, a concise sparse Cholesky package}
\author{Timothy A. Davis\thanks{
email: DrTimothyAldenDavis@gmail.com,
http://www.suitesparse.com.
This work was supported by the National
Science Foundation, under grant CCR-0203270.
Portions of the work were done while on sabbatical at Stanford University
and Lawrence Berkeley National Laboratory (with funding from Stanford
University and the SciDAC program).
}}

\date{VERSION 2.2.3, Jan 30 2016}

\maketitle

%-------------------------------------------------------------------------------
\begin{abstract}
The {\tt LDL} software package is a set of short, concise routines for
factorizing symmetric positive-definite sparse matrices, with some
applicability to symmetric indefinite matrices.  Its primary purpose is
to illustrate much of the basic theory of sparse matrix algorithms in as
concise a code as possible, including an elegant method
of sparse symmetric factorization that computes the factorization row-by-row
but stores it column-by-column.  The entire symbolic and numeric factorization
consists of less than 50 lines of code.  The package is written in C,
and includes a MATLAB interface.
\end{abstract}
%-------------------------------------------------------------------------------

%-------------------------------------------------------------------------------
\section{Overview}
%-------------------------------------------------------------------------------

{\tt LDL} is a set of short, concise routines that compute the $\m{LDL}\tr$
factorization of a sparse symmetric matrix $\m{A}$.  Its primary purpose is
to illustrate much of the basic theory of sparse matrix algorithms in as
compact a code as possible, including an elegant method of
sparse symmetric factorization (related to \cite{Liu86c,Liu91}).
The lower triangular factor $\m{L}$ is computed row-by-row, in contrast to the
conventional column-by-column method.
Although it does not achieve the same level of performance
as methods based on dense matrix kernels
(such as \cite{NgPeyton93,RothbergGupta91}),
its performance is competitive with column-by-column methods that do not
use dense kernels \cite{GeorgeLiu79, GeorgeLiu, GilbertMolerSchreiber}.

Section~\ref{Algorithm} gives a brief description of the algorithm
used in the symbolic and numeric factorization.  A more detailed tutorial-level
discussion may be found in \cite{Stewart03}.  Details
of the concise implementation of this method are given in
Section~\ref{Implementation}.  Sections~\ref{MATLAB}~and~\ref{C} give an
overview of how to use the package in MATLAB and in a stand-alone C program.

%-------------------------------------------------------------------------------
\section{Algorithm}
\label{Algorithm}
%-------------------------------------------------------------------------------

The underlying numerical algorithm is described below.  The $k$th
step solves a lower triangular system of dimension $k-1$ to compute the
$k$th row of $\m{L}$ and the $d_{kk}$ entry of the diagonal matrix $\m{D}$.
Colon notation is used for submatrices.  For example,
$\m{L}_{k,1:k-1}$ refers to the first $k-1$ columns of
the $k$th row of $\m{L}$.  Similarly, $\m{L}_{1:k-1,1:k-1}$ refers to
the leading $(k-1)$-by-$(k-1)$ submatrix of $\m{L}$.
%---------------
\vspace{-0.2in}
\begin{tabbing}
\hspace{2em} \= \hspace{2em} \= \hspace{2em} \= \\
{\bf Algorithm~1
($\m{LDL}\tr$ factorization of a $n$-by-$n$ symmetric matrix $\m{A}$)} \\
\> {\bf for} $k = 1$ {\bf to} $n$ \\
\>\> (step 1) Solve $\m{L}_{1:k-1,1:k-1}\m{y} = \m{A}_{1:k-1,k}$ for $\m{y}$ \\
\>\> (step 2) $\m{L}_{k,1:k-1} = (\m{D}_{1:k-1,1:k-1}^{-1} \m{y})\tr$ \\
\>\> (step 3) $l_{kk} = 1$ \\
\>\> (step 4) $d_{kk} = a_{kk} - \m{L}_{k,1:k-1}\m{y}$ \\
\> {\bf end for}
\end{tabbing}
%---------------

The algorithm computes an $\m{LDL}\tr$ factorization without numerical pivoting.
It can thus factorize any symmetric positive definite matrix, and any
symmetric indefinite matrix whose leading minors are all well-conditioned.

When $\m{A}$ and $\m{L}$ are sparse, step 1 of Algorithm~1 requires a
triangular solve of the form $\m{Lx}=\m{b}$, where all three terms in
the equation are sparse.  This is the most costly step of the Algorithm.
Steps 2 through 4 are fairly straightforward.

Let ${\cal X}$ and ${\cal B}$ refer to the set of indices of nonzero entries
in $\m{x}$ and $\m{b}$, respectively, in the lower triangular system
$\m{Lx}=\m{b}$.  To compute $\m{x}$ efficiently
the nonzero pattern ${\cal X}$ must be found first.
In the general case when $\m{L}$ is arbitrary \cite{GilbertPeierls88},
the nonzero
pattern ${\cal X}$ is the set of nodes reachable via paths in the graph $G_L$
from all nodes in the set ${\cal B}$, and where the graph $G_L$ has
$n$ nodes and a directed edge $(j,i)$ if and only if $l_{ij}$ is nonzero.
To compute the numerical solution to $\m{Lx}=\m{b}$ by accessing the columns of
$\m{L}$ one at a time, ${\cal X}$ can be traversed
in any topological order of the subgraph of $G_L$ consisting of nodes in
${\cal X}$.  That is, $x_j$ must be computed before $x_i$ if there is a path
from $j$ to $i$ in $G_L$.  The natural order ($1, 2, \ldots, n$) is one such
ordering, but that requires a costly sort of ${\cal X}$.
With a graph traversal and topological sort, the solution of $\m{Lx}=\m{b}$
can be computed using Algorithm~2 below.
The computation of ${\cal X}$ and $\m{x}$ both take
time proportional to the floating-point operation count.
%---------------
\vspace{-0.2in}
\begin{tabbing}
\hspace{2em} \= \hspace{2em} \= \hspace{2em} \= \\
{\bf Algorithm~2
(Solve $\m{Lx}=\m{b}$, where $\m{L}$ is lower triangular with unit diagonal)} \\
\> ${\cal X} = \mbox{Reach}_{G_L} ({\cal B})$ \\
\> $\m{x} = \m{b}$ \\
\> {\bf for} $i \in {\cal X}$ in any topological order \\
\>\> $\m{x}_{i+1:n} = \m{x}_{i+1:n} - \m{L}_{i+1:n,i} x_i$ \\
\> {\bf end for}
\end{tabbing}
%---------------

The general result also governs the pattern of $\m{y}$ in Algorithm~1.
However, in this case $\m{L}$ arises from a sparse Cholesky factorization,
and is governed by the elimination tree \cite{Liu90a}.
A general graph traversal is not required.
In the elimination tree, the parent of node $i$ is the smallest $j > i$
such that $l_{ji}$ is nonzero.  Node $i$ has no parent if column $i$ of
$\m{L}$ is completely zero below the diagonal; $i$ is a root of the
elimination tree in this case.  The nonzero pattern of $\m{x}$ is the
union of all the nodes on the paths from any node $i$ (where $b_i$ is nonzero) to the
root of the elimination tree \cite[Thm 2.4]{Liu86c}.  It is referred to here as a tree,
but in general it can be a forest.

Rather than a general topological sort of the subgraph of $G_L$ consisting
nodes reachable from nodes in ${\cal B}$, a simpler
tree traversal can be used.  First, select any nonzero entry $b_i$
and follow the path from $i$ to the root of tree.
Nodes along this path are marked and placed in a stack,
with $i$ at the top of the
stack and the root at the bottom.
Repeat for every other nonzero entry in $b_i$, in arbitrary order, but stop
just before reaching a marked node (the result can be empty if $i$ is already
in the stack).  The stack now contains ${\cal X}$, a topological ordering of
the nonzero pattern of $\m{x}$, which can be used in Algorithm~2 to solve
$\m{Lx}=\m{b}$.  The time to compute ${\cal X}$
using an elimination tree traversal is much faster than the general graph
traversal, taking time proportional to the size of ${\cal X}$ rather than the
number of floating-point operations required to compute $\m{x}$.

In the $k$th step of the factorization, the set ${\cal X}$ becomes the
nonzero pattern of row $k$ of $\m{L}$.  This step requires the elimination
tree of $\m{L}_{1:k-1,1:k-1}$, and must construct the elimination tree of
$\m{L}_{1:k,1:k}$ for step $k+1$.  Recall that the parent of $i$ in the
tree is the smallest $j$ such that $i < j$ and $l_{ji} \ne 0$.
Thus, if any node $i$ already has a parent $j$, then $j$ will remain the
parent of $i$ in the elimination trees of all other larger leading submatrices
of $\m{L}$, and in the elimination tree of $\m{L}$ itself.
If $l_{ki} \ne 0$ and $i$ does not have a parent in the elimination tree of
$\m{L}_{1:k-1,1:k-1}$, then the parent of $i$ is $k$
in the elimination tree of $\m{L}_{1:k,1:k}$.
Node $k$ becomes the parent of any node $i \in {\cal X}$ that does not yet
have a parent.

Since Algorithm~2 traverses $\m{L}$ in column order, $\m{L}$ is stored in a
conventional sparse column representation.  Each column $j$ is stored as a list
of nonzero values and their corresponding row indices.  When row $k$ is
computed, the new entries can be placed at the end of each list.  As
a by-product of computing $\m{L}$ one row at a time,
the columns of $\m{L}$ are computed in a sorted manner.  This is a convenient
form of the output.
MATLAB requires the columns of its sparse matrices to be sorted, for example.
Sorted columns improve the speed of Algorithm~2, since the memory access
pattern is more regular.  The conventional column-by-column algorithm
\cite{GeorgeLiu79,GeorgeLiu} does not produce columns of $\m{L}$ with
sorted row indices.

A simple symbolic pre-analysis can be obtained by repeating the subtree traversals.
All that is required to compute the nonzero pattern of
the $k$th row of $\m{L}$ is the partially constructed elimination tree
and the nonzero pattern of the $k$th column of $\m{A}$.  This is computed
in time proportional to the size of this set, using the elimination tree
traversal.  Once constructed, the number of nonzeros in each column of
$\m{L}$ is incremented, for each entry in ${\cal X}$, and then ${\cal X}$
is discarded.  The set ${\cal X}$ need not be constructed in topological
order, so no stack is required.  The run time of the symbolic analysis
algorithm is thus proportional to the number of nonzeros in $\m{L}$.
This is more costly than the optimal algorithm \cite{GilbertNgPeyton94},
which takes time essentially proportional to the number of nonzeros in $\m{A}$.
The memory requirements are just the matrix $\m{A}$ and a few size-$n$ integer
arrays.  The result of the algorithm is the elimination tree, a count
of the number of nonzeros in each column of $\m{L}$, and
the cumulative sum of the column counts.

%-------------------------------------------------------------------------------
\section{Implementation}
\label{Implementation}
%-------------------------------------------------------------------------------

Because of its simplicity, the implementation of this algorithm leads to
a very short, concise code.  The symbolic analysis routine {\tt ldl\_symbolic}
shown in Figure~\ref{ldlsymbolic}
consists of only 18 lines of executable C code.
This includes 5 lines of code to allow for a
sparsity-preserving ordering $\m{P}$ so that either $\m{A}$ or $\m{PAP}\tr$
can be analyzed, 3 lines of code to compute the cumulative sum of
the column counts, and one line of code to speed up a {\tt for} loop.
An additional line of code allows for a more general form of the input
sparse matrix $\m{A}$.

The {\tt n}-by-{\tt n} sparse matrix $\m{A}$ is provided in compressed column
form as an {\tt int} array {\tt Ap} of length {\tt n+1},
an {\tt int} array {\tt Ai} of length {\tt nz},
and a {\tt double} array {\tt Ax} also of length {\tt nz},
where {\tt nz} is the number of entries in the matrix.
The numerical values of entries in column $j$ are stored in
{\tt Ax[Ap[j]} $\ldots$ {\tt Ap[j+1]-1]}
and the corresponding row indices are in
{\tt Ai[Ap[j]} $\ldots$ {\tt Ap[j+1]-1]}.
With {\tt Ap[0] = 0}, the number of entries in the matrix is {\tt nz = Ap[n]}.
If no fill-reducing ordering {\tt P} is provided,
only entries in the upper triangular part of $\m{A}$ are considered.
If {\tt P} is provided and row/column {\tt i} of the
matrix $\m{A}$ is the {\tt k}-th row/column of $\m{PAP}\tr$, then {\tt P[k]=i}.
Only entries in the upper
triangular part of $\m{PAP}\tr$ are considered.  These entries may be
in the lower triangular part of $\m{A}$, so to ensure that the correct matrix
is factorized, all entries of $\m{A}$ should be provided when using the
permutation input {\tt P}.

The outputs of {\tt ldl\_symbolic} are three size-{\tt n} arrays:
{\tt Parent} holds the elimination tree,
{\tt Lnz} holds the counts of the number of entries in each column of
$\m{L}$, and
{\tt Lp} holds the cumulative sum of {\tt Lnz}.
The size-{\tt n} array {\tt Flag} is used as workspace.
None of the output or workspace arrays need to be initialized.

\begin{figure}
\caption{{\tt ldl\_symbolic:} finding the elimination tree and column counts}
\label{ldlsymbolic}
{\scriptsize
\begin{verbatim}
void ldl_symbolic
(
    int n,              /* A and L are n-by-n, where n >= 0 */
    int Ap [ ],         /* input of size n+1, not modified */
    int Ai [ ],         /* input of size nz=Ap[n], not modified */
    int Lp [ ],         /* output of size n+1, not defined on input */
    int Parent [ ],     /* output of size n, not defined on input */
    int Lnz [ ],        /* output of size n, not defined on input */
    int Flag [ ],       /* workspace of size n, not defn. on input or output */
    int P [ ],          /* optional input of size n */
    int Pinv [ ]        /* optional output of size n (used if P is not NULL) */
)
{
    int i, k, p, kk, p2 ;
    if (P)
    {
        /* If P is present then compute Pinv, the inverse of P */
        for (k = 0 ; k < n ; k++)
        {
            Pinv [P [k]] = k ;
        }
    }
    for (k = 0 ; k < n ; k++)
    {
        /* L(k,:) pattern: all nodes reachable in etree from nz in A(0:k-1,k) */
        Parent [k] = -1 ;           /* parent of k is not yet known */
        Flag [k] = k ;              /* mark node k as visited */
        Lnz [k] = 0 ;               /* count of nonzeros in column k of L */
        kk = (P) ? (P [k]) : (k) ;  /* kth original, or permuted, column */
        p2 = Ap [kk+1] ;
        for (p = Ap [kk] ; p < p2 ; p++)
        {
            /* A (i,k) is nonzero (original or permuted A) */
            i = (Pinv) ? (Pinv [Ai [p]]) : (Ai [p]) ;
            if (i < k)
            {
                /* follow path from i to root of etree, stop at flagged node */
                for ( ; Flag [i] != k ; i = Parent [i])
                {
                    /* find parent of i if not yet determined */
                    if (Parent [i] == -1) Parent [i] = k ;
                    Lnz [i]++ ;                         /* L (k,i) is nonzero */
                    Flag [i] = k ;                      /* mark i as visited */
                }
            }
        }
    }
    /* construct Lp index array from Lnz column counts */
    Lp [0] = 0 ;
    for (k = 0 ; k < n ; k++)
    {
        Lp [k+1] = Lp [k] + Lnz [k] ;
    }
}
\end{verbatim}
}
\end{figure}

The {\tt ldl\_numeric} numeric factorization routine shown
in Figure~\ref{ldlnumeric} consists of only 31 lines of
executable code.  It includes this same subtree traversal algorithm
as {\tt ldl\_symbolic},
except that each path is placed on a stack that holds
nonzero pattern of the $k$th row of $\m{L}$.
This traversal is followed by a sparse forward solve
using this pattern, and all of the nonzero entries in
the resulting $k$th row of $\m{L}$ are appended to their respective columns
in the data structure of $\m{L}$.

\begin{figure}
\caption{{\tt ldl\_numeric:} numeric factorization}
\label{ldlnumeric}
{\scriptsize
\begin{verbatim}
int ldl_numeric         /* returns n if successful, k if D (k,k) is zero */
(
    int n,              /* A and L are n-by-n, where n >= 0 */
    int Ap [ ],         /* input of size n+1, not modified */
    int Ai [ ],         /* input of size nz=Ap[n], not modified */
    double Ax [ ],      /* input of size nz=Ap[n], not modified */
    int Lp [ ],         /* input of size n+1, not modified */
    int Parent [ ],     /* input of size n, not modified */
    int Lnz [ ],        /* output of size n, not defn. on input */
    int Li [ ],         /* output of size lnz=Lp[n], not defined on input */
    double Lx [ ],      /* output of size lnz=Lp[n], not defined on input */
    double D [ ],       /* output of size n, not defined on input */
    double Y [ ],       /* workspace of size n, not defn. on input or output */
    int Pattern [ ],    /* workspace of size n, not defn. on input or output */
    int Flag [ ],       /* workspace of size n, not defn. on input or output */
    int P [ ],          /* optional input of size n */
    int Pinv [ ]        /* optional input of size n */
)
{
    double yi, l_ki ;
    int i, k, p, kk, p2, len, top ;
    for (k = 0 ; k < n ; k++)
    {
        /* compute nonzero Pattern of kth row of L, in topological order */
        Y [k] = 0.0 ;               /* Y(0:k) is now all zero */
        top = n ;                   /* stack for pattern is empty */
        Flag [k] = k ;              /* mark node k as visited */
        Lnz [k] = 0 ;               /* count of nonzeros in column k of L */
        kk = (P) ? (P [k]) : (k) ;  /* kth original, or permuted, column */
        p2 = Ap [kk+1] ;
        for (p = Ap [kk] ; p < p2 ; p++)
        {
            i = (Pinv) ? (Pinv [Ai [p]]) : (Ai [p]) ;   /* get A(i,k) */
            if (i <= k)
            {
                Y [i] += Ax [p] ;  /* scatter A(i,k) into Y (sum duplicates) */
                for (len = 0 ; Flag [i] != k ; i = Parent [i])
                {
                    Pattern [len++] = i ;   /* L(k,i) is nonzero */
                    Flag [i] = k ;          /* mark i as visited */
                }
                while (len > 0) Pattern [--top] = Pattern [--len] ;
            }
        }
        /* compute numerical values kth row of L (a sparse triangular solve) */
        D [k] = Y [k] ;             /* get D(k,k) and clear Y(k) */
        Y [k] = 0.0 ;
        for ( ; top < n ; top++)
        {
            i = Pattern [top] ;     /* Pattern [top:n-1] is pattern of L(:,k) */
            yi = Y [i] ;            /* get and clear Y(i) */
            Y [i] = 0.0 ;
            p2 = Lp [i] + Lnz [i] ;
            for (p = Lp [i] ; p < p2 ; p++)
            {
                Y [Li [p]] -= Lx [p] * yi ;
            }
            l_ki = yi / D [i] ;     /* the nonzero entry L(k,i) */
            D [k] -= l_ki * yi ;
            Li [p] = k ;            /* store L(k,i) in column form of L */
            Lx [p] = l_ki ;
            Lnz [i]++ ;             /* increment count of nonzeros in col i */
        }
        if (D [k] == 0.0) return (k) ;      /* failure, D(k,k) is zero */
    }
    return (n) ;        /* success, diagonal of D is all nonzero */
}
\end{verbatim}
}
\end{figure}

After the matrix is factorized, the {\tt ldl\_lsolve}, {\tt ldl\_dsolve},
and {\tt ldl\_ltsolve} routines shown in Figure~\ref{ldlsolve}
are provided to solve
$\m{Lx}=\m{b}$, $\m{Dx}=\m{b}$, and $\m{L}\tr\m{x}=\m{b}$, respectively.
Together, they solve $\m{Ax}=\m{b}$, and consist of only 10 lines of executable
code.  If a fill-reducing permutation is used,
{\tt ldl\_perm} and {\tt ldl\_permt} must be used to permute $\m{b}$ and
$\m{x}$ accordingly.

\begin{figure}
\caption{Solve routines}
\label{ldlsolve}
{\scriptsize
\begin{verbatim}
void ldl_lsolve
(
    int n,              /* L is n-by-n, where n >= 0 */
    double X [ ],       /* size n.  right-hand-side on input, soln. on output */
    int Lp [ ],         /* input of size n+1, not modified */
    int Li [ ],         /* input of size lnz=Lp[n], not modified */
    double Lx [ ]       /* input of size lnz=Lp[n], not modified */
)
{
    int j, p, p2 ;
    for (j = 0 ; j < n ; j++)
    {
        p2 = Lp [j+1] ;
        for (p = Lp [j] ; p < p2 ; p++)
        {
            X [Li [p]] -= Lx [p] * X [j] ;
        }
    }
}

void ldl_dsolve
(
    int n,              /* D is n-by-n, where n >= 0 */
    double X [ ],       /* size n.  right-hand-side on input, soln. on output */
    double D [ ]        /* input of size n, not modified */
)
{
    int j ;
    for (j = 0 ; j < n ; j++)
    {
        X [j] /= D [j] ;
    }
}

void ldl_ltsolve
(
    int n,              /* L is n-by-n, where n >= 0 */
    double X [ ],       /* size n.  right-hand-side on input, soln. on output */
    int Lp [ ],         /* input of size n+1, not modified */
    int Li [ ],         /* input of size lnz=Lp[n], not modified */
    double Lx [ ]       /* input of size lnz=Lp[n], not modified */
)
{
    int j, p, p2 ;
    for (j = n-1 ; j >= 0 ; j--)
    {
        p2 = Lp [j+1] ;
        for (p = Lp [j] ; p < p2 ; p++)
        {
            X [j] -= Lx [p] * X [Li [p]] ;
        }
    }
}
\end{verbatim}
}
\end{figure}

In addition to appearing as a Collected Algorithm of the ACM \cite{Davis05},
{\tt LDL} is available at http://www.suitesparse.com.

%-------------------------------------------------------------------------------
\section{Using LDL in MATLAB}
\label{MATLAB}
%-------------------------------------------------------------------------------

The simplest way to use {\tt LDL} is within MATLAB.  Once the {\tt ldlsparse}
mexFunction is compiled and installed, the MATLAB statement
{\tt [L, D, Parent, fl] = ldlsparse (A)} returns the sparse factorization
{\tt A = (L+I)*D*(L+I)'}, where {\tt L} is lower triangular, {\tt D} is a
diagonal matrix, and {\tt I} is the {\tt n}-by-{\tt n}
identity matrix ({\tt ldlsparse} does not return the unit diagonal of {\tt L}).
The elimination tree is returned in {\tt Parent}.
If no zero on the diagonal of {\tt D} is encountered, {\tt fl} is the
floating-point operation count.  Otherwise, {\tt D(-fl,-fl)} is the first
zero entry encountered.  Let {\tt d=-fl}.  The function returns the
factorization of {\tt A (1:d,1:d)}, where rows {\tt d+1} to {\tt n} of {\tt L}
and {\tt D} are all zero.  If a sparsity preserving permutation {\tt P} is
passed, {\tt [L, D, Parent, fl] = ldlsparse (A,P)}
operates on {\tt A(P,P)} without
forming it explicitly.

The statement {\tt x = ldlsparse (A, [ ], b)} is roughly equivalent to
{\tt x = A}$\backslash${\tt b}, when {\tt A} is sparse, real, and symmetric.
The $\m{LDL}\tr$ factorization of {\tt A} is performed.  If {\tt P} is
provided, {\tt x = ldlsparse (A, P, b)} still performs
{\tt x = A}$\backslash${\tt b}, except that {\tt A(P,P)} is factorized
instead.

%-------------------------------------------------------------------------------
\section{Using LDL in a C program}
\label{C}
%-------------------------------------------------------------------------------

To compile the library, do {\tt make}.  To install the shared library in
/usr/local/include and /usr/local/lib, do {\tt make install}; to remove it
from there, use {\tt make uninstall}.
For alternative installation locations, see the instructions in
{\tt SuiteSparse/README.txt}.

The C-callable {\tt LDL} library consists of nine user-callable routines
and one include file.

\begin{itemize}
\item {\tt ldl\_symbolic}:  given the nonzero pattern of a sparse symmetric
    matrix $\m{A}$ and an optional permutation $\m{P}$, analyzes either
    $\m{A}$ or $\m{PAP}\tr$, and returns the elimination tree, the
    number of nonzeros in each column of $\m{L}$, and the {\tt Lp} array
    for the sparse matrix data structure for $\m{L}$.
    Duplicate entries are allowed in the columns of $\m{A}$, and the
    row indices in each column need not be sorted.
    Providing a sparsity-preserving ordering is critical for obtaining
    good performance.  A minimum degree ordering
    (such as AMD \cite{AmestoyDavisDuff96,AmestoyDavisDuff03})
    or a graph-partitioning based ordering are appropriate.
\item {\tt ldl\_numeric}:  given {\tt Lp} and the elimination tree computed
    by {\tt ldl\_symbolic}, and an optional permutation $\m{P}$,
    returns the numerical factorization of $\m{A}$ or $\m{PAP}\tr$.
    Duplicate entries are allowed in the columns of $\m{A}$
    (any duplicate entries are summed), and the
    row indices in each column need not be sorted.
    The data structure for $\m{L}$ is the same as $\m{A}$, except that
    no duplicates appear, and each column has sorted row indices.
\item {\tt ldl\_lsolve}:  given the factor $\m{L}$ computed by
    {\tt ldl\_numeric}, solves the linear system $\m{Lx}=\m{b}$, where
    $\m{x}$ and $\m{b}$ are full $n$-by-1 vectors.
\item {\tt ldl\_dsolve}:  given the factor $\m{D}$ computed by
    {\tt ldl\_numeric}, solves the linear system $\m{Dx}=\m{b}$.
\item {\tt ldl\_ltsolve}:  given the factor $\m{L}$ computed by
    {\tt ldl\_numeric}, solves the linear system $\m{L}\tr\m{x}=\m{b}$.
\item {\tt ldl\_perm}: given a vector $\m{b}$ and a permutation $\m{P}$,
    returns $\m{x}=\m{Pb}$.
\item {\tt ldl\_permt}: given a vector $\m{b}$ and a permutation $\m{P}$,
    returns $\m{x}=\m{P}\tr\m{b}$.
\item {\tt ldl\_valid\_perm}:  Except for checking if the diagonal of
    $\m{D}$ is zero, none of the above routines check their inputs for errors.
    This routine checks the validity of a permutation $\m{P}$.
\item {\tt ldl\_valid\_matrix}:  checks if a matrix $\m{A}$ is valid as input
    to {\tt ldl\_symbolic} and {\tt ldl\_numeric}.
\end{itemize}

Note that the primary input to the {\tt ldl\_symbolic} and
{\tt ldl\_numeric} is the sparse matrix $\m{A}$.  It is provided in
column-oriented form, and only the upper triangular part is accessed.
This is slightly different than the primary output: the matrix $\m{L}$, 
which is lower triangular in column-oriented form.
If you wish to factorize a symmetric matrix $\m{A}$ for which only the lower
triangular part is supplied, you would need to transpose $\m{A}$ before
passing it {\tt ldl\_symbolic} and {\tt ldl\_numeric}.

An additional set of routines is available for use in a 64-bit environment.
Each routine name changes uniformly; {\tt ldl\_symbolic} becomes
{\tt ldl\_l\_symbolic}, and each {\tt int} parameter becomes type
{\tt SuiteSparse\_long}.  The {\tt SuiteSparse\_long} type is {\tt long}, except for
Microsoft Windows 64, where it becomes {\tt \_\_int64}.

\begin{figure}
\caption{Example of use}
\label{ldlsimple}
{\scriptsize
\begin{verbatim}
#include <stdio.h>
#include "ldl.h"
#define N 10    /* A is 10-by-10 */
#define ANZ 19  /* # of nonzeros on diagonal and upper triangular part of A */
#define LNZ 13  /* # of nonzeros below the diagonal of L */

int main (void)
{
    /* only the upper triangular part of A is required */
    int    Ap [N+1] = {0, 1, 2, 3, 4,   6, 7,   9,   11,      15,     ANZ},
           Ai [ANZ] = {0, 1, 2, 3, 1,4, 5, 4,6, 4,7, 0,4,7,8, 1,4,6,9 } ;
    double Ax [ANZ] = {1.7, 1., 1.5, 1.1, .02,2.6, 1.2, .16,1.3, .09,1.6,
                     .13,.52,.11,1.4, .01,.53,.56,3.1},
           b [N] = {.287, .22, .45, .44, 2.486, .72, 1.55, 1.424, 1.621, 3.759};
    double Lx [LNZ], D [N], Y [N] ;
    int Li [LNZ], Lp [N+1], Parent [N], Lnz [N], Flag [N], Pattern [N], d, i ;

    /* factorize A into LDL' (P and Pinv not used) */
    ldl_symbolic (N, Ap, Ai, Lp, Parent, Lnz, Flag, NULL, NULL) ;
    printf ("Nonzeros in L, excluding diagonal: %d\n", Lp [N]) ;
    d = ldl_numeric (N, Ap, Ai, Ax, Lp, Parent, Lnz, Li, Lx, D, Y, Pattern,
        Flag, NULL, NULL) ;

    if (d == N)
    {
        /* solve Ax=b, overwriting b with the solution x */
        ldl_lsolve (N, b, Lp, Li, Lx) ;
        ldl_dsolve (N, b, D) ;
        ldl_ltsolve (N, b, Lp, Li, Lx) ;
        for (i = 0 ; i < N ; i++) printf ("x [%d] = %g\n", i, b [i]) ;
    }
    else
    {
        printf ("ldl_numeric failed, D (%d,%d) is zero\n", d, d) ;
    }
    return (0) ;
}
\end{verbatim}
}
\end{figure}

The program in Figure~\ref{ldlsimple}
illustrates the basic usage of the {\tt LDL} routines.
It analyzes and factorizes the sparse symmetric positive-definite matrix
{\small
\[
\m{A} = \left[
\begin{array}{cccccccccc}
      1.7 &   0 &   0 &   0 &   0 &   0 &   0 &   0 & .13 &   0 \\
        0 &  1. &   0 &   0 & .02 &   0 &   0 &   0 &   0 & .01 \\
        0 &   0 & 1.5 &   0 &   0 &   0 &   0 &   0 &   0 &   0 \\
        0 &   0 &   0 & 1.1 &   0 &   0 &   0 &   0 &   0 &   0 \\
        0 & .02 &   0 &   0 & 2.6 &   0 & .16 & .09 & .52 & .53 \\
        0 &   0 &   0 &   0 &   0 & 1.2 &   0 &   0 &   0 &   0 \\
        0 &   0 &   0 &   0 & .16 &   0 & 1.3 &   0 &   0 & .56 \\
        0 &   0 &   0 &   0 & .09 &   0 &   0 & 1.6 & .11 &   0 \\
      .13 &   0 &   0 &   0 & .52 &   0 &   0 & .11 & 1.4 &   0 \\
        0 & .01 &   0 &   0 & .53 &   0 & .56 &   0 &   0 & 3.1 \\
\end{array}
\right]
\]
}
and then solves a system $\m{Ax}=\m{b}$ whose true solution is
$x_i = i/10$.  Note that {\tt Li} and {\tt Lx} are statically allocated.
Normally they would be allocated after their size, {\tt Lp[n]},
is determined by {\tt ldl\_symbolic}.
More example programs are included with the {\tt LDL} package.

\section{Acknowledgments}

I would like to thank Pete Stewart for his comments on an earlier draft
of this software and its accompanying paper.

\newpage
\bibliographystyle{plain}
\bibliography{ldl}

\end{document}
