%-------------------------------------------------------------------------------
% The PIRO\_BAND/Doc/UserGuide.tex file.
%-------------------------------------------------------------------------------

\documentclass[11pt]{article}

\newcommand{\m}[1]{{\bf{#1}}}       % for matrices and vectors
\newcommand{\tr}{^{\sf T}}          % transpose
\newcommand{\new}[1]{\overline{#1}}

\topmargin 0in
\textheight 9in
\oddsidemargin 0pt
\evensidemargin 0pt
\textwidth 6.5in

\newenvironment{packed_itemize}{
\begin{itemize}
  \setlength{\itemsep}{1pt}
  \setlength{\parskip}{0pt}
  \setlength{\parsep}{0pt}
}{\end{itemize}}

\begin{document}

\author{Sivasankaran Rajamanickam, Timothy A. Davis \\
Dept. of Computer and Information Science and Engineering \\
Univ. of Florida, Gainesville, FL}
\title{User Guide for PIRO\_BAND: General band reduction using 
blocked Givens rotations}
\date{Version 1.0, May 9, 2014}
\maketitle

%-------------------------------------------------------------------------------
\begin{abstract}
    PIRO\_BAND is package for reducing both symmetric and unsymmetric band
    matrices to tridiagonal and bidiagonal matrices respectively using blocked 
    and pipelined Givens rotations. The package also includes MATLAB interfaces
    for finding the Singular Value Decomposition of a band matrix using the
    bidiagonal reduction algorithm. It also supports the exact interface
    for LAPACK's band reduction routines. PIRO\_BAND is written in ANSI/ISO C.
    It is tested on various Unix variants and Microsoft Windows. The packages
    can handle both real and complex matrices. The package is competitive with
    other band reduction packages.
\end{abstract}
%-------------------------------------------------------------------------------

PIRO\_BAND Copyright\copyright 2009-2014 by Sivasankaran Rajamanickam and 
Timothy A. Davis.  

All Rights Reserved.  PIRO\_BAND's Modules are distributed under the GNU
General Public License.
PIRO\_BAND is also available under other licenses that permit its use in
proprietary applications; contact the authors for details.
See http://www.cise.ufl.edu/research/sparse for the code and all documentation,
including this User Guide.

\newpage
\tableofcontents

%-------------------------------------------------------------------------------
\newpage \section{Overview}
%-------------------------------------------------------------------------------

PIRO\_BAND is a package for reducing banded matrices $\m{A}$ such that 

\begin{center} $\m{A} = \m{U} * \m{B} * \m{V'}$ \end{center}

where $\m{B}$ is a bidiagonal matrix when $\m{A}$ is an unsymmetric matrix and
$\m{B}$ is a tridiagonal matrix when A is a symmetric matrix. $\m{U}$ and
$\m{V}$ are orthonormal matrices. PIRO\_BAND uses blocked and pipelined Givens
rotations to reduce the band matrices. By pipelining PIRO\_BAND will not
generate more than a scalar fill.  Blocking the Givens rotations enables
PIRO\_BAND to be cache efficient.

PIRO\_BAND provides one common interface for reducing matrices that are either
symmetric or unsymmetric, real or complex with 32-bit or 64-bit integer
dimensions. The package is tested on various UNIX variants, in both 32 bit and
64 bit architectures, and Microsoft Windows. The C library also supports
LAPACK's eight different interfaces for band reduction too. 

The MATLAB interface supports band reduction to bidiagonal or tridiagonal form,
and a function for computing the singular value decomposition of a sparse
matrix that exploits the band.  The MATLAB interface supports both full and
sparse matrices.

%-------------------------------------------------------------------------------
\newpage \section{Compiling PIRO\_BAND}
%-------------------------------------------------------------------------------

PIRO\_BAND requires {\tt make} to compile the C callable libraries and the test
coverage. The MATLAB mex files can be compiled with {\tt make} or from within
MATLAB itself.  Further test coverage assumes C99 style complex data type,
but this is not assumed in the rest of the package.

Obtain and install SuiteSparse\_config and edit the configuration file in the
{\tt SuiteSparse\_config/SuiteSparse\_config.mk} file.  Sample configurations
are provided for Linux, Macintosh, and many other systems.

\noindent
Here are the various parameters that you can control in your
{\tt SuiteSparse\_config/SuiteSparse\_config.mk}
file to compile and install PIRO\_BAND :
\begin{itemize}
\item {\tt CC = } your C compiler, such as {\tt cc}.
\item {\tt CFLAGS = } optimization flags, such as {\tt -O}.
\item {\tt RANLIB = } your system's {\tt ranlib} program, if needed.
\item {\tt AR =} the command to create a library (such as {\tt ar}).
\item {\tt RM =} the command to delete a file.
\item {\tt MV =} the command to rename a file.
\item {\tt LIB = } basic libraries, such as {\tt -lm}.
\item {\tt MEX =} the command to compile a MATLAB mexFunction.
\end{itemize}

%-------------------------------------------------------------------------------
\subsection{Compiling the C-Callable library}
%-------------------------------------------------------------------------------

Run {\tt make} in the {\tt PIRO\_BAND} directory. This will compile the
PIRO\_BAND library and a demo code in the {\tt PIRO\_BAND/Demo} directory ({\tt
test\_piro\_band}).
To run the simple test code run
{\tt test\_piro\_band} in the directory {\tt PIRO\_BAND/Test}.  To run the
entire suite of tests that cover all the lines in the code type {\tt make cov}
in the directory {\tt PIRO\_BAND/Tcov}. The norms that are printed should be
small.  PIRO\_BAND library is now ready to use in your own applications. 

You can include the header files {\tt piro\_band.h} and {\tt
piro\_band\_lapack.h} in your code to use the PIRO\_BAND's interface or LAPACK
style interface in your code. The header files are in {\tt PIRO\_BAND/Include}
directory. To compile your application successfully include {\tt
PIRO\_BAND/Include} in your application's include path and link to {\tt
PIRO\_BAND/Source/libpiro\_band.a} with your own code.   Alternatively,
{\tt make install} will copy this files to {\tt /usr/local/lib}
and {\tt /usr/local/include}.

To compile only the C-callable libraries type {\tt make} in {\tt
PIRO\_BAND/Source} directory, or {\tt make library} in the {\tt PIRO\_BAND}
directory.  This will generate {\tt PIRO\_BAND/Source/libpiro\_band.a} but will
not compile the test code, test coverage or the MATLAB mex files.

%-------------------------------------------------------------------------------
\subsection{Compiling and Installing PIRO\_BAND for use in MATLAB }
%-------------------------------------------------------------------------------

PIRO\_BAND provides a MATLAB interface for the band reduction and a
singular value decomposition for sparse or full matrices that exploits
the band.

To compile the MATLAB mexFunctions, type {\tt piro\_band\_make} while in the
{\tt PIRO\_BAND/MATLAB} directory.  This also updates your path for the current
MATLAB session.  Running {\tt testall} in the {\tt PIRO\_BAND/MATLAB} directory
runs an exhaustive test.  All the norms printed should be small. Your matlab
libraries and m-files are ready to use.

%-------------------------------------------------------------------------------
\newpage \section{Using PIRO\_BAND in MATLAB}
%-------------------------------------------------------------------------------

PIRO\_BAND provides various mex functions and m-files to use in MATLAB. The
following is a list of functions with a short decription.

\vspace{0.1in}
\begin{tabular}{ll}
\hline
{\tt piro\_band}         & Bidiagonal reduction routine for band matrices \\
{\tt piro\_band\_svd}    & SVD of a band matrix \\
\end{tabular}

The usage for each of the function follows.

\subsection{{\tt piro\_band} : Bidiagonal reduction of band matrices}
\input{_piro_band_m.tex}

\subsection{{\tt piro\_band\_svd} : Singular value Decompostion of band matrices}
\input{_piro_band_svd_m.tex}

%-------------------------------------------------------------------------------
\newpage \section{Using PIRO\_BAND C library}
%-------------------------------------------------------------------------------

Functions in PIRO\_BAND C library can be used by including one of two header
files in your application {\tt PIRO\_BAND/Include/piro\_band.h} if you use
PIRO\_BAND interface for the band reduction or {\tt
PIRO\_BAND/Include/piro\_band\_lapack.h} if you prefer to use the LAPACK style
of interface. In general we recommend to use our interface as it avoids two
transposes. You have to include
{\tt SuiteSparse\_config/SuiteSparse\_config.mk}
too. The code will look like

\begin{verbatim}
#include "SuiteSparse_config.h"
#include "piro\_band.h"
\end{verbatim}

or 

\begin{verbatim}
#include "SuiteSparse_config.h"
#include "piro\_band_lapack.h"
\end{verbatim}

The primary function for band reduction in PIRO\_BAND is {\tt
piro\_band\_reduce} which has eight different versions as described in section
\ref{reduce_all}. You can call any of these versions based on the required
precision, architecture, and whether the inputs are real or complex.

If you wish to know a good  blocksize you can call the function {\tt
piro\_band\_get\_blocksize} to get a recommended block size which you may then
pass to one of the {\tt reduce} functions.

When your application starts, and prior to calling any PIRO\_BAND function (or
any other function in SuiteSparse), you should call SuiteSparse\_start.  This
function sets various global function pointers, for malloc, calloc, realloc,
free, printf, timing routines, and basic mathematics utility functions.  The
function modifies a globally-accessible struct, so it is not thread-safe.  All
threads use the same set of functions for these operations, so no single thread
should have its own copy anyway.  As of SuiteSparse 4.3.0, calling this
function is optional, but this may change in future versions.  There is also a
corresponding SuiteSparse\_finish, which you should call when your entire
application exists.  Currently, the function is an empty placeholder, but
future versions may change this.

%-------------------------------------------------------------------------------
\subsection{PIRO\_BAND Naming conventions and parameters}
%-------------------------------------------------------------------------------

All the function names provided by PIRO\_BAND has the prefix {\tt
piro\_band\_}.  There are two different styles for the suffixes. A suffix of
the style {\tt \_xyz} where {\tt x}, {\tt y} and {\tt z} should be replaced by
the appropiate letters given below to get the actual function name. 

\begin{itemize}
\item {\tt x =  d | s} for double or single precision respectively.
\item {\tt y =  r | c} for real or complex matrices respectively.
\item {\tt z =  i | l} for 32-bit or 64-bit integers.
\end{itemize}

For example, the primary function to reduce a band matrix to bidiagoal or 
tridiagonal form is {\tt piro\_band\_reduce\_xyz} allows all the eight
combinations resulting in the following functions. All the prototypes are
listed in section \ref{reduce_all}.

\vspace{0.1in}
\begin{tabular}{ll}
\hline
{\tt piro\_band\_reduce\_dri} & for double precision, real matrices and 32 bit integers \\
{\tt piro\_band\_reduce\_drl} & for double precision, real matrices and 64 bit integers \\
{\tt piro\_band\_reduce\_sri} & for single precision, real matrices and 32 bit integers \\
{\tt piro\_band\_reduce\_srl} & for single precision, real matrices and 64 bit integers \\
{\tt piro\_band\_reduce\_dci} & for double precision, complex matrices and 32 bit integers \\
{\tt piro\_band\_reduce\_dcl} & for double precision, complex matrices and 64 bit integers \\
{\tt piro\_band\_reduce\_sci} & for single precision, complex matrices and 32 bit integers \\
{\tt piro\_band\_reduce\_scl} & for single precision, complex matrices and 64 bit integers \\
\end{tabular}

For functions that differ only in the usage of 32-bit and 64-bit integers the
suffix {\tt \_l} is used for 64-bit versions while 32-bit version do not have
any suffix. For example, the function to get the recommended blocksize has two
versions {\tt piro\_band\_get\_blocksize} and {\tt
piro\_band\_get\_blocksize\_l} for 32-bit and 64-bit versions respectively.

PIRO\_BAND functions do not differentiate between symmetric and unsymmetric
matrices at the interface level. Instead we use a parameter to the function to
differentiate between them.

All the functions in the LAPACK style interface have the prefix {\tt
piro\_band\_} added to the original LAPACK names. They may have the suffix {\tt
\_l} added to them depending on whether it is the 32-bit or 64-bit version. For
example, LAPACK's bidiagonal reduction routine {\tt dgbbrd} is called {\tt
piro\_band\_dgbbrd} or {\tt piro\_band\_dgbbrd\_l} in our LAPACK style
interface.

The return value of zero means success.  Error return codes are
are described in Section \ref{err_code}.

%-------------------------------------------------------------------------------
\subsection{Workspace requirements}
%-------------------------------------------------------------------------------

PIRO\_BAND requires a floating point work space to store the blocked Givens
rotations. The size of the work space depends on the block size which is a user
controlled parameter : the first parameter in all the eight versions of the
{\tt piro\_band\_reduce} functions. The block size parameter is an array of
size four where the first two entries specify the number of columns and rows in
the block to reduce the band in the upper triangular part. The next two entries
specify the number of rows and columns in the block to reduce the lower
triangular part.  The workspace required is twice the maximum blocksize. For
example for a 10-by-10 matrix with both upper and lower bandwidth 5 the code
snippet to allocate the workspace looks like

\begin{verbatim}
    int blks[4] ;
    int msize ;
    double *wspace ;

    piro\_band_get_blocksize(10, 10, 5, 5, blks) ;
    msize = 2 * MAX(blks[0]*blks[1], blks[2]*blks[3]) ;
    wspace = (double *) malloc(msize * sizeof(double)) ;
\end{verbatim}

If the input matrices are complex then the work space required will be twice
the work space required for reducing real matrices. If you pass a NULL block
size and NULL work space then PIRO\_BAND will determine the best block size and
allocate the required work space internally.

%-------------------------------------------------------------------------------
\subsection{PIRO\_BAND interface for bidiagonal reduction of band matrices}
\label{reduce_all}
%-------------------------------------------------------------------------------

This section lists the prototypes for the eight different versions of the
{\tt piro\_band\_reduce} function.

%-------------------------------------------------------------------------------
\subsubsection{{\tt piro\_band\_reduce\_dri}} \label{reduce}
%-------------------------------------------------------------------------------

\input{_piro_band_reduce_1.tex}
To reduce a real symmetric or unsymmetric band matrix to a tridiagonal or
bidiagonal matrix using double precision arithmetic and 32-bit integers.

The other functions that are similar in the funtionality except for the data
types are :

\subsubsection{{\tt piro\_band\_reduce\_drl}}
\input{_piro_band_reduce_2.tex}
To reduce a real symmetric or unsymmetric band matrix to a
tridiagonal or bidiagonal matrix using double precision arithmetic and 64-bit
integers. See section \ref{reduce} for detailed description of the parameters.

\subsubsection{{\tt piro\_band\_reduce\_sri}}
\input{_piro_band_reduce_3.tex}
To reduce a real symmetric or unsymmetric band matrix to a
tridiagonal or bidiagonal matrix using single precision arithmetic and 
32-bit integers. See section \ref{reduce} for detailed description of the parameters.

\subsubsection{{\tt piro\_band\_reduce\_srl}}
\input{_piro_band_reduce_4.tex}
To reduce a real symmetric or unsymmetric band matrix to a
tridiagonal or bidiagonal matrix using single precision arithmetic and 64-bit
integers. See section \ref{reduce} for detailed description of the parameters.

\subsubsection{{\tt piro\_band\_reduce\_dci}}
\input{_piro_band_reduce_3.tex}
To reduce a complex symmetric or unsymmetric band matrix to a
tridiagonal or bidiagonal matrix using double precision arithmetic and 
32-bit integers. See section \ref{reduce} for detailed description of the parameters.

\subsubsection{{\tt piro\_band\_reduce\_dcl}}
\input{_piro_band_reduce_2.tex}
To reduce a complex symmetric or unsymmetric band matrix to a
tridiagonal or bidiagonal matrix using double precision arithmetic and 64-bit
integers. See section \ref{reduce} for detailed description of the parameters.

\subsubsection{{\tt piro\_band\_reduce\_sci}}
\input{_piro_band_reduce_3.tex}
To reduce a complex symmetric or unsymmetric band matrix to a
tridiagonal or bidiagonal matrix using single precision arithmetic and 
32-bit integers. See section \ref{reduce} for detailed description of the parameters.

\subsubsection{{\tt piro\_band\_reduce\_scl}}
\input{_piro_band_reduce_4.tex}
To reduce a complex symmetric or unsymmetric band matrix to a
tridiagonal or bidiagonal matrix using single precision arithmetic and 64-bit
integers. See section \ref{reduce} for detailed description of the parameters.

%-------------------------------------------------------------------------------
\subsection{Recommened block size for bidiagonal reduction}
%-------------------------------------------------------------------------------

\subsubsection{{\tt piro\_band\_get\_blocksize}}
\input{_piro_band_blk_1.tex}
To get the recommended block size for a given matrix in 32-bit architectures.

\subsubsection{{\tt piro\_band\_get\_blocksize\_l}}
\input{_piro_band_blk_2.tex}
To get the recommended block size for a given matrix in 64-bit architectures.

%-------------------------------------------------------------------------------
\subsection{LAPACK style interface for bidiagonal reduction of unsymmetric band matrices}
%-------------------------------------------------------------------------------

\subsubsection{{\tt piro\_band\_dgbbrd}}
\input{_piro_band_lapack_1.tex}
To reduce a real unsymmetric band matrix to a bidiagonal matrix using double 
precision arithmetic and 32-bit integers.

\subsubsection{{\tt piro\_band\_dgbbrd\_l}}
\input{_piro_band_lapack_2.tex}
To reduce a real unsymmetric band matrix to a bidiagonal matrix using double 
precision arithmetic and 64-bit integers.

\subsubsection{{\tt piro\_band\_zgbbrd}}
\input{_piro_band_lapack_3.tex}
To reduce a complex unsymmetric band matrix to a bidiagonal matrix using double 
precision arithmetic and 32-bit integers.

\subsubsection{{\tt piro\_band\_zgbbrd\_l}}
\input{_piro_band_lapack_4.tex}
To reduce a complex unsymmetric band matrix to a bidiagonal matrix using double 
precision arithmetic and 64-bit integers.

\subsubsection{{\tt piro\_band\_sgbbrd}}
\input{_piro_band_lapack_5.tex}
To reduce a real unsymmetric band matrix to a bidiagonal matrix using single 
precision arithmetic and 32-bit integers.

\subsubsection{{\tt piro\_band\_sgbbrd\_l}}
\input{_piro_band_lapack_6.tex}
To reduce a real unsymmetric band matrix to a bidiagonal matrix using single 
precision arithmetic and 64-bit integers.

\subsubsection{{\tt piro\_band\_cgbbrd}}
\input{_piro_band_lapack_7.tex}
To reduce a complex unsymmetric band matrix to a bidiagonal matrix using single 
precision arithmetic and 32-bit integers.

\subsubsection{{\tt piro\_band\_cgbbrd\_l}}
\input{_piro_band_lapack_8.tex}
To reduce a complex unsymmetric band matrix to a bidiagonal matrix using single 
precision arithmetic and 64-bit integers.

%-------------------------------------------------------------------------------
\subsection{LAPACK style interface for bidiagonal reduction of symmetric band matrices}
%-------------------------------------------------------------------------------

\subsubsection{{\tt piro\_band\_dsbtrd}}
\input{_piro_band_lapack_9.tex}
To reduce a real symmetric band matrix to a bidiagonal matrix using double 
precision arithmetic and 32-bit integers.

\subsubsection{{\tt piro\_band\_dsbtrd\_l}}
\input{_piro_band_lapack_10.tex}
To reduce a real symmetric band matrix to a bidiagonal matrix using double 
precision arithmetic and 64-bit integers.

\subsubsection{{\tt piro\_band\_zhbtrd}}
\input{_piro_band_lapack_11.tex}
To reduce a hermitian band matrix to a bidiagonal matrix using double 
precision arithmetic and 32-bit integers.

\subsubsection{{\tt piro\_band\_zhbtrd\_l}}
\input{_piro_band_lapack_12.tex}
To reduce a hermitian band matrix to a bidiagonal matrix using double 
precision arithmetic and 64-bit integers.

\subsubsection{{\tt piro\_band\_ssbtrd}}
\input{_piro_band_lapack_13.tex}
To reduce a real symmetric band matrix to a bidiagonal matrix using single 
precision arithmetic and 32-bit integers.

\subsubsection{{\tt piro\_band\_ssbtrd\_l}}
\input{_piro_band_lapack_14.tex}
To reduce a real symmetric band matrix to a bidiagonal matrix using single 
precision arithmetic and 64-bit integers.

\subsubsection{{\tt piro\_band\_chbtrd}}
\input{_piro_band_lapack_15.tex}
To reduce a hermitian band matrix to a bidiagonal matrix using single 
precision arithmetic and 32-bit integers.

\subsubsection{{\tt piro\_band\_chbtrd\_l}}
\input{_piro_band_lapack_16.tex}
To reduce a hermitian band matrix to a bidiagonal matrix using single 
precision arithmetic and 64-bit integers.

%-------------------------------------------------------------------------------
\subsection{Error codes from PIRO\_BAND}
\label{err_code}
%-------------------------------------------------------------------------------

Return values are described below.  The return value of zero means success.
The first set of error codes match those codes returned by LAPACK.

\vspace{0.1in}
\begin{tabular}{lll}
\hline
{\tt PIRO\_BAND\_OK}               & 0   & success \\
{\tt PIRO\_BAND\_VECT\_INVALID}    & -1  & {\tt VECT} input is not valid \\
{\tt PIRO\_BAND\_M\_INVALID}       & -2  & {\tt M} input is not valid \\
{\tt PIRO\_BAND\_N\_INVALID}       & -3  & {\tt N} input is not valid \\
{\tt PIRO\_BAND\_NRC\_INVALID}     & -4  & {\tt NRC} input is not valid \\
{\tt PIRO\_BAND\_BL\_INVALID}      & -5  & {\tt BL} input is not valid \\
{\tt PIRO\_BAND\_BU\_INVALID}      & -6  & {\tt BU} input is not valid \\
{\tt PIRO\_BAND\_LDAB\_INVALID}    & -8  & {\tt LDAB} input is not valid \\
{\tt PIRO\_BAND\_LDU\_INVALID}     & -12 & {\tt LDU} input is not valid \\
{\tt PIRO\_BAND\_LDV\_INVALID}     & -14 & {\tt LDV} input is not valid \\
{\tt PIRO\_BAND\_LDC\_INVALID}     & -16 & {\tt LDC} input is not valid \\
\hline
{\tt PIRO\_BAND\_A\_INVALID}       & -7  & {\tt A} is a NULL pointer \\
{\tt PIRO\_BAND\_B1\_INVALID}      & -9  & {\tt B1} is a NULL pointer \\
{\tt PIRO\_BAND\_B2\_INVALID}      & -10 & {\tt B2} is a NULL pointer \\
{\tt PIRO\_BAND\_U\_INVALID}       & -11 & {\tt U} is a NULL pointer \\
{\tt PIRO\_BAND\_V\_INVALID}       & -13 & {\tt V} is a NULL pointer \\
{\tt PIRO\_BAND\_C\_INVALID}       & -15 & {\tt C} is a NULL pointer \\
{\tt PIRO\_BAND\_SYM\_INVALID}     & -17 & {\tt SYM} input is not valid \\
{\tt PIRO\_BAND\_BLKSIZE\_INVALID} & -18 & {\tt BLKS} input is not valid \\
%
% Error codes for LAPACK style routines
%
{\tt PIRO\_BAND\_OUT\_OF\_MEMORY}  & -19 & out of memory for work space \\
{\tt PIRO\_BAND\_UPLO\_INVALID}    & -20 & {\tt UPLO} input is not valid \\
%
% error codes originating from calls to LAPACK
{\tt PIRO\_BAND\_LAPACK\_INVALID}  & -21 & internal error \\
{\tt PIRO\_BAND\_LAPACK\_FAILURE}  & -22 & SVD did not converge \\
%
\hline
\end{tabular}

%-------------------------------------------------------------------------------
\newpage \section{PIRO\_BAND interface vs LAPACK style interface}
%-------------------------------------------------------------------------------

PIRO\_BAND supports two interfaces:  a native one, and one that is compatible
with LAPACK.  They differ in four ways, listed below.  The first two simplify
the algorithm for band reduction.  The last two are for efficient computation.

\begin{enumerate}
\item
PIRO\_BAND interface requires the upper bandwidth to be at least one. The
LAPACK style interface adds a zero diagonal to upper triangular part if the
upper bandwidth is zero.

\item
For symmetric matrices PIRO\_BAND requires the upper triangular part to be
stored. The lower triangular part may or may not be stored. The LAPACK style
interface transposes the input matrix and the results if only the lower
triangular part is stored for a symmtric matrix.

\item
PIRO\_BAND finds $\m{C}'\m{Q}$ instead of $\m{Q}'\m{C}$.  The former is more
efficient, but the latter is the LAPACK standard.  The LAPACK style interface
uses two transposes for compatibility. 

\item
PIRO\_BAND finds $\m{V}$ rather than the $\m{V'}$ matrix that LAPACK computes.
As in the previous cases the LAPACK style interface transposes to
return $\m{V'}$ if required, for compatibility with LAPACK.
\end{enumerate}

\end{document}
